\documentclass[10pt,a4paper]{article}
 
\usepackage{amsmath}
\usepackage{listings}
\usepackage{hyperref}
\usepackage[thinc]{esdiff}

\renewcommand{\thesection}{Question \arabic{section}}
\renewcommand{\thesubsection}{Part \arabic{section}.\alph{subsection}}


\title{Project: Symbolic Computation}
\author{Name \\ \url{email@u.nus.edu.sg}} 

\begin{document}

\maketitle

	
\section{}
We consider the polynomial ring.

\subsection{}
Write a definite clause grammar that  recognises and parses polynomials with real coefficients. We use the LaTeX mathematical notation (see \url{https://www.overleaf.com/learn/latex/List_of_Greek_letters_and_math_symbols}) to input and output the polynomials and the the letter \verb+x+ to represent the indeterminate. The polynomial of Equation~\ref{eq:equation1} is rendered polynomial for the LaTex code: \verb&[5.2 \times  x^{22}- 3.44 \times x^3+25&.

\begin{equation} 
\label{eq:equation1}
5.2 \times  x^{22}- 3.44 \times x^3+25
\end{equation}



Indicate in you report which additional shorthands and notations (e.g. associativity, recognising $5.2 \  x^{22}$ as $5.2   \times x^{22}$) you implement.
\subsection{}
 Some Prolog code is presented in Figure~\ref{fig:code1}.


\begin{figure}

\begin{lstlisting}[language=Prolog,frame = single,basicstyle=\footnotesize,numbers=left,stepnumber=1]
% member/2
member(X,[X|R]).
member(X,[Y|R]) :- member(X,R).
\end{lstlisting}
\caption{Prolog code snippet (Figures are floating)}
\label{fig:code1}
\end{figure}
\section{} You may download, intall and use a free and open-source TeX front-end program, like TeXworks (\url{http://www.tug.org/texworks/}), or an online LaTeX editor like Overleaf (\url{https://www.overleaf.com/}) to edit and process the report.

\end{document}


